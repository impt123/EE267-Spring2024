% \documentclass[11pt]{article}
% \usepackage[utf8]{inputenc}	% Para caracteres en español
% \usepackage{amsmath,amsthm,amsfonts,amssymb,amscd}
% \usepackage{multirow,booktabs}
% \usepackage[table]{xcolor}
% \usepackage{fullpage}
% \usepackage{lastpage}
% \usepackage{enumitem}
% \usepackage{fancyhdr}
% \usepackage{mathrsfs}
% \usepackage{wrapfig}
% \usepackage{setspace}
% \usepackage{calc}
% \usepackage{multicol}
% \usepackage{cancel}
% \usepackage[retainorgcmds]{IEEEtrantools}
% \usepackage[margin=3cm]{geometry}
% \usepackage{amsmath}
% \newlength{\tabcont}
% \setlength{\parindent}{0.0in}
% \setlength{\parskip}{0.05in}
% \usepackage{empheq}
% \usepackage{framed}
% \usepackage[most]{tcolorbox}
% \usepackage{xcolor}
% \colorlet{shadecolor}{orange!15}
% \geometry{margin=1in, headsep=0.25in}

\documentclass[11pt]{article}
 
\usepackage[margin=1in]{geometry} 
\usepackage{amsmath,amsthm,amssymb,scrextend}
\usepackage{fancyhdr}
\usepackage{graphicx}
\setlength{\headheight}{14pt}
\usepackage{xcolor}
\usepackage{color,soul}
\usepackage{gensymb}
\pagestyle{fancy}


\begin{document}
 
% --------------------------------------------------------------
%                         Start here
% --------------------------------------------------------------

\lhead{EE267 Virtual Reality}
\chead{Homework 5 Answers}
\rhead{Bohan Li}

\section{Theoretical Part}
\subsection{Rotation Quaternions}
The rotation quaternion with the rotation angle as $\theta$ and rotation axis $\textbf{v} = (v_x, v_y, v_z)^T $ is given as:
\begin{align}
    q(\theta,v) = \cos(\theta/2) + i v_x \sin(\theta/2) + j v_y \sin(\theta/2) + k v_z \sin(\theta/2).
\end{align}
The length of $q$ is \[|q| = \sqrt{\cos^2(\theta/2)+( v_x^2 + v_y^2 + v_z^2) \sin^2(\theta/2)} = \cos^2(\theta/2) + \sin^2(\theta/2) = 1.\]

\subsection{3-D Gyro Integration}
\subsubsection*{(i)}
The refresh rate is $1\ Hz$ so the time step is given as $1\ s$. The angle of rotation is given as $||\omega||\Delta{t}$ and the rotation axis is $\frac{\omega}{||\omega||}$. Therefore, the rotation axis and amount of rotation for each 1, 2, 3, 4 is:
\begin{align*}
    & ||\theta^{(1)}|| = 90 \degree,\ axis^{(1)}=(1,0,0)^T\\
    & ||\theta^{(2)}|| = -90 \degree,\ axis^{(2)}=(0,0,1)^T\\
    & ||\theta^{(3)}|| = -90 \degree,\ axis^{(3)}=(0,1,0)^T\\
    & ||\theta^{(4)}|| = 90 \degree,\ axis^{(4)}=(0,0,1)^T
\end{align*}
Or,
\begin{enumerate}
    \item Rotate around x-axis by $90 \degree$;
    \item Rotate around z-axis by $-90 \degree$;
    \item Rotate around y-axis by $-90 \degree$;
    \item Rotate around z-axis by $90 \degree$;
\end{enumerate}

\subsubsection*{(ii)}
Following the procedure summarized in (i), we get the full rotation as: 
\begin{align}
    R_{tot} = &R_z(\pi/2)R_y(-\pi/2)R_z(-\pi/2)R_x(\pi/2) \\
    & = \begin{pmatrix}0&-1&0\\1&0&0\\0&0&1\end{pmatrix} \begin{pmatrix}0&0&-1\\0&1&0\\1&0&0\end{pmatrix} \begin{pmatrix}0&1&0\\-1&0&0\\0&0&1\end{pmatrix} \begin{pmatrix}1&0&0\\0&0&-1\\0&1&0\end{pmatrix}\\
    & = \begin{pmatrix}1&0&0\\0&-1&0\\0&0&-1\end{pmatrix} 
\end{align}
So the sequence of operation is equivalent of getting the inverse of the y- and z- coordinates. 

\subsubsection*{(iii)}
Following the first-order Taylor series, the angle at each time point is:
\[
    \theta^{(0)}=(0,0,0)^T\to\theta^{(1)}=(\frac{\pi}{2},0,0)^T\to\theta^{(2)}=(\frac{\pi}{2},0,-\frac{\pi}{2})^T\to\theta^{(3)}=(\frac{\pi}{2},-\frac{\pi}{2},-\frac{\pi}{2})^T\to\theta^{(4)}=(\frac{\pi}{2},-\frac{\pi}{2},0)^T 
\]
To verify, the total rotation for Euler angle is given as:
\begin{equation}
    R = R_z(\theta_z)R_x(-\theta_x)R_y(-\theta_y). 
\end{equation}
Therefore the rotation matrix for the Euler angle for each step is:
\begin{equation}
    R^{(1)} = \begin{pmatrix}1&0&0\\0&0&1\\0&-1&0\end{pmatrix}  
\end{equation}
\begin{equation}
    R^{(2)} = \begin{pmatrix}0&0&-1\\-1&0&0\\0&-1&0\end{pmatrix}  
\end{equation}
\begin{equation}
    R^{(3)} = \begin{pmatrix}1&0&0\\0&0&-1\\0&1&0\end{pmatrix}  
\end{equation}
\begin{equation}
    R^{(4)} = \begin{pmatrix}0&0&-1\\-1&0&0\\0&1&0\end{pmatrix}  
\end{equation}
The resulting R4 is not the same as the rotation matrix in (ii).

\subsubsection*{(iv)}
Due to the non-commute nature of rotation operation, the actual procedure matters for rotation. The effect of a sequence of rotation cannot by equivalently represented as a full rotation given as the final outcome angle (Euler angle). 
From the operational perspective, the Euler angle is illy defined when the vector is along certain axis around which the rotation happens, introducing ambiguity. 

\newpage

\section*{Programming Part}
\subsection*{}


\end{document}